\documentclass[onecolumn,superscriptaddress,notitlepage]{revtex4}
%%%%%%%%%%%%%%%%%%%%%%%%%%%%%%%%%%%%%%%%%%%%%%%%%%%%%%%%%%%%%%%%%%%%%%%%%%%%%%%%%%%%%%%%%%%%%%%%%%%%%%%%%%%
\usepackage{amssymb}
\usepackage{amsmath}
\usepackage{graphicx}
\usepackage{epstopdf}
\usepackage{subfigure}
\usepackage{natbib}
\usepackage{epsfig}
\usepackage{amsfonts}
\usepackage{mathrsfs}
\usepackage[toc,page,title,titletoc,header]{appendix}
\usepackage[colorlinks,linkcolor=blue,citecolor=blue,anchorcolor=blue]{hyperref}
\usepackage{dsfont,amsthm,amsbsy}
\usepackage{titlesec}
\usepackage{indentfirst}

\makeatletter
\renewcommand{\section}{\@startsection{section}{1}{0mm}{-\baselineskip}{0.5\baselineskip}{\bf}}
\renewcommand\thesection{\arabic{section}}
\makeatother

\begin{document}
\title{Floquet quantum system in a classical bath}

\author{Bingtian Ye}
\affiliation{Department of Physics, Univerisity of California, Berkeley}
\author{Marin Bukov}
\affiliation{Department of Physics, Univerisity of California, Berkeley}
\maketitle
\section{the Hamiltonian}
The full Hamiltonian is seperated into three parts:
\begin{equation}
\hat{H}=\hat{H}_{q}+\hat{H}_{c}+\hat{H}_{int},
\end{equation}
where $\hat{H}_{q}$, $\hat{H}_{c}$, and $\hat{H}_{int}$ are the Hamiltonian for the quantum part, classical part and interaction respectively. 
All the three terms can be formally expressed as:
\begin{equation}
\hat{H}_{q,c,int}=
\begin{cases}
\sum_{i,j} J_{i,j} S^z_i S^z_j + \sum_i g_i S^z_i & nT\le t<(n+1/2)T \\
\sum_i h_i S^x_i & (n+1/2)T\le t<(n+1)T,
\end{cases}
\end{equation}
where $S^\alpha_i$ represents a pauli matrix $\hat{\sigma}^\alpha_i$ for a quantum spin on site $i$, and it represents a 3D unit vector $\vec{s}^\alpha_i$ for a classical spin on site $i$. 

$\hat{H}_{q}$ only consists of quantum operators, and has no ambiguity when it acts on quantum spins. 
$\hat{H}_{c}$ determines the evolution of classical spins by Hamilton canonical equations and Poisson brackets:
\begin{gather}
\frac{ds^\alpha_i}{dt} = \{s^\alpha_i,H_c\}\\
\{s^\alpha_i,s^\beta_j\} = \delta_{i,j} \epsilon^{\alpha\beta\gamma}s^\gamma_i,
\end{gather}
As for the interaction $\hat{H}_{int}$, we treat the classical spin $\vec{S}$ as time-dependent local fields when apply it to quantum spins, and use the expectation value of quantum spins $\langle\hat{S}\rangle$ as time-dependent local fields for the related classical spins. 

All the three terms preserve $S^z$ during the first half of driving period, and preserve $S^x$ during the second half, which simplifies the simulation a lot. 
For the quantum part, we can apply local unitary evolution on each bond in sequence, without introducing Trotter errors. 
For the classical part, we can derive an analytical expression for the evolution. 
In particular, during the first half of period, the equations of motion are:
\begin{equation}
\begin{split}
\frac{ds^x_i}{dt} &= -\tilde{g}_i s^y_i, \\
\frac{ds^y_i}{dt} &= \tilde{g}_i s^x_i, \\
\frac{ds^z_i}{dt} &= 0, 
\end{split}
\end{equation}
where $\tilde{g}_i = g_i+\sum_{j} J_{i,j} S^z_j$ with $j$ running over all sites adjacent to $i$. 
Therefore, the time evolution from $nT$ to $(n+1)T$ for any site is a rotation:
\begin{equation}
\tau_1(\vec{s}_i)=
\begin{pmatrix}
s^x_i \mathrm{cos}(\tilde{g_i}T/2)-s^y_i \mathrm{sin}(\tilde{g_i}T/2)\\
s^x_i \mathrm{sin}(\tilde{g_i}T/2)+s^y_i \mathrm{cos}(\tilde{g_i}T/2)\\
s^z_i
\end{pmatrix},
\end{equation}
in which the values of $s^\alpha_i$ are chosen at $t=nT$. 
Similarly, the evolution from $(n+1/2)T$ to $(n+1)T$ can be expressed as:
\begin{equation}
\tau_2(\vec{s}_i)=
\begin{pmatrix}
s^x_i\\
s^y_i \mathrm{cos}(h_i T/2)-s^z_i \mathrm{sin}(h_i T/2)\\
s^y_i \mathrm{sin}(h_i T/2)+s^z_i \mathrm{cos}(h_i  T/2)
\end{pmatrix},
\end{equation}
in which the values of $s^\alpha_i$ are chosen at $t=(n+1/2)T$. 
Then the full evolution during a driving period is given by the map $\tau_2 \circ \tau_1$.

\end{document}